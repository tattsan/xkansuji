%%%%%%%%%%%%%%%%%%%%%%%%%%%%%%%%%%%%%%%%%%%%%%%%%%%%%%%%%%%%%%%%%%
\documentclass[dvipdfmx]{jsarticle}
\usepackage{otf}\PassOptionsToPackage{otf}{xkansuji}
\usepackage[moga-mobo,directunicode]{pxchfon}
\usepackage{ascmac}
%%%%%%%%%%%%%%%%%%%%%%%%%%%%%%%%%%%%%%%%%%%%%%%%%%%%%%%%%%%%%%%%%%
% \documentclass[dvipdfmx,uplatex]{jsarticle}
% \usepackage[uplatex]{otf}\PassOptionsToPackage{otf}{xkansuji}
% \usepackage[moga-mobo,directunicode]{pxchfon}
% \usepackage{ascmac}
%%%%%%%%%%%%%%%%%%%%%%%%%%%%%%%%%%%%%%%%%%%%%%%%%%%%%%%%%%%%%%%%%
% \documentclass{ltjsarticle}
% \usepackage{luatexja-fontspec}
% \setmainjfont{MogaMincho}\setsansjfont{MogaGothic}
% \usepackage{bxascmac}
%%%%%%%%%%%%%%%%%%%%%%%%%%%%%%%%%%%%%%%%%%%%%%%%%%%%%%%%%%%%%%%%%%
% \documentclass[xelatex]{bxjsarticle}
% \usepackage{zxjatype}
% \xeCJKDeclareCharClass{CJK}{`■}
% \setmainfont{MogaMincho}\setCJKmainfont{MogaMincho}
% \setCJKsansfont{MogaGothic}\setCJKmonofont{MogaGothic}
% \usepackage{bxascmac}
%%%%%%%%%%%%%%%%%%%%%%%%%%%%%%%%%%%%%%%%%%%%%%%%%%%%%%%%%%%%%%%%%%

%%%%%%%%%%%%%%%%%%%%%%%%%%%%%%%%%%%%%%%%%%%%%%%%%%%%%%%%%%%%%%%%%%
\usepackage[dvipsnames]{color}\PassOptionsToPackage{color}{xkansuji}
%%%%%%%%%%%%%%%%%%%%%%%%%%%%%%%%%%%%%%%%%%%%%%%%%%%%%%%%%%%%%%%%%%

\usepackage{xkansuji}
\makeatletter
\def\JOcuurentChar{\xksj@Jo@command}%
\begingroup
\SetJOjp \expandafter\gdef\expandafter\JOjpChar\expandafter{\xksj@Jo@command}%
\SetJOcn \expandafter\gdef\expandafter\JOcnChar\expandafter{\xksj@Jo@command}%
\endgroup
\def\YUjpChar{\xksj@YU}
\makeatother
\usepackage{amsmath}
\usepackage{multicol}
\usepackage{xint}
\newif\ifSkipXint
% \SkipXinttrue % xint での計算が長引く場合に、計算済みの値を利用する。
\SkipXintfalse % xint での計算が長引いても、xintで全部計算する。



\title{\textsf{Xkansujiパッケージ}}
\author{tattsan}
\date{}
\begin{document}
\maketitle
漢数字の表示マクロを作ったら明後日の方向に進んでゆき、実に役に立たないパッケージになった。
\def\thesection{第\xkansuji{section}話}
\setcounter{section}{\xintiSub{\xintiPow{2}{31}}{4}}%% 2^{31}-4
\section{普通の命数法}
当初は普通の漢数字マクロだった。
p\TeX\space のプリミティブで \verb+\kansuji2014年+ が二〇一四年と表示されるのを、
\Xkansuji{2014}年に変えることが目的のはずだった。
\begin{screen}
\begin{itemize}
\item \verb+\xkansuji+ \LaTeX のカウンタの表示
\item \verb+\@xkansuji+ \TeX カウンタ(\LaTeX カウンタの内部表現など)の表示
\item \verb+\Xkansuji+ アラビア数字の文字列を変換して表示
\end{itemize}  
\end{screen}
このようなマクロは既に多くの人が手掛けているから、何か特色を出したいと思い
\textbf{大きな数を表示できること}を心掛けた。
たとえば \TeX のカウンタが扱える範囲は $1-2^{31}$ から $2^{31}-1$ までであるから、
\[
2^{31}-1=\xintiSub{\xintiPow{2}{31}}{1}
\]
は当然表示できねばならない。
なお本パッケージの機能は単に数を表示するだけなので、
$2^{31}$の十進展開を「計算する」ことは出来ない。
この文書では計算については \textsf{xintパッケージ} を利用している。
例えばセクションの番号の設定は、このセクションに入る前に次のように入力している
\footnote{%
\texttt{\textbackslash xintiSub}の所は以前は\texttt{\textbackslash xintSub}だった。
\texttt{\textbackslash xintSub}
は現在の\textsf{xintパッケージ}では有理数計算に用いられ、出力形式が異なる。}。
\begin{verbatim}
\def\thesection{第\xkansuji{section}回}
\setcounter{section}{\xintiSub{\xintiPow{2}{31}}{4}}%% 2^{31}-4
\end{verbatim}

\paragraph{「\JOjpChar」について。}
一,十,百,千,万,億,兆,京,垓,\JOjpChar,穣,溝, $\cdots$ の
% 「\JOjpChar」%𥝱 = utf8:^^f0^^a5^^9d^^b1 = \UTF{25771} = \CID{17018}
「\UTF{25771}」%𥝱 = utf8:^^f0^^a5^^9d^^b1 = \UTF{25771} = \CID{17018}
という文字について。この文字はJIS第2水準には存在せず、Unicode の第0面(基本多言語面, BMP)
にも存在しない。Unicode の基本面にも無いのは、これが元々は
「\JOcnChar」(U+79ED, CID 17022)%秭 = utf8:^^e7^^a7^^ad = \UTF {79ED} = \CID{17022}
という文字であり、それが誤写によって形を変えた国字だからである。
OTFフォントなら「\JOjpChar」は CID 17018 でアクセスできる。
Unicodeなら第2面(追加漢字面, SIP)のU+25771にある。
\begin{itemize}
\item パッケージ・オプション\texttt{otf}で、\textsf{otfパッケージ}の \verb+\CID{17018}+ を用いて「\JOjpChar」を表示する。
  \TeX エンジンの自動認識には未対応なので \verb+\RequirePackage+ はしていない。
  このため文書ファイル側で適切なパッケージを \verb+\usepackage+ する必要がある。
\item \texttt{otf}オプションを用いない場合、直接入力で追加漢字面を表示できる必要がある。
\item \verb+\SetJOjp+ で「\JOjpChar」が、\verb+\SetJOcn+ で「\JOcnChar」が用いられる。
\end{itemize}

\newpage
\noindent ここでちょっと表示を試してみよう。
\verb+\[2^{100}=\text{\Xkansuji{\xintiPow{2}{100}}}\]+ の結果は
\small
\[
2^{100}=\text{\Xkansuji{\xintiPow{2}{100}}}
\]
\normalsize
である。
\paragraph{数詞が不足する問題}
そこでもう少し大きな数を表示させてみよう。たとえば次を入力してみると

\hfil\verb+$1234567^{89}=$\Xkansuji{\xintiPow{1234567}{89}}+\bigskip

$1234567^{89}=$\Xkansuji{\xintiPow{1234567}{89}}


\noindent
\textbf{残念ながら、あっさりと数詞が足りなくなってしまう。}
「無量大数」なんて大した数ではないことがわかる。

\section{上数法による表示}
漢数字による通常の表示方法は古来「中数」と呼ばれている。
この他「下数」「上数」という命数法がある\footnote{%
「中数」において「一万倍ごとに」単位を考える方法を「万進」と呼ぶ。中数には「万進」以外に
「一億倍ごとに」単位を考える「万万進」がある。「億以上はすべて万万進」「恒河沙以上のみ万万進」
などいろいろあるので、ここでは割愛する。}。
「万」までは共通だが、そのあとの決め方が異なる。「下数」は
\begin{itembox}[l]{下数法}
\[
\text{千}
\ \xrightarrow{\quad\text{十倍}\quad}\ {}
\text{万}
\ \xrightarrow{\quad\text{十倍}\quad}\ {}
\text{億}
\ \xrightarrow{\quad\text{十倍}\quad}\ {}
\text{兆}
\ \xrightarrow{\quad\text{十倍}\quad}\ {}
\text{京}
\ \xrightarrow{\quad\text{十倍}\quad}\ {}\cdots
\]
\end{itembox}
のように十倍ごとに数の名称を変える方法である。
古い時代にはこのような命数法が用いられたようだ。

一方「上数」は、たとえば「億」の「十万倍」は「十万億」と呼び、
「億億=兆」のように本当に言葉が足りなくなるごとに(二乗ごとに)新しい数詞を導入する
方法である。このルールは「万」以上に適用され、「千」までは下数法と共通とする。
\begin{itembox}[l]{上数法}
\[
\text{万}
\ \xrightarrow{\quad\text{万倍}\quad}\ {}
\text{億}
\ \xrightarrow{\quad\text{億倍}\quad}\ {}
\text{兆}
\ \xrightarrow{\quad\text{兆倍}\quad}\ {}
\text{京}
\ \xrightarrow{\quad\text{京倍}\quad}\ {}
\text{垓}
\ \xrightarrow{\quad\text{垓倍}\quad}\ {}\ {}\cdots
\]
\end{itembox}
指数表記を用いて、数の表現能力を比較してみよう。
機械的に計算すると次の表のようになる。
\begin{center}
  \begin{tabular}{|c|c|c|c|c|c|c|c|c|c|c|c|c|}\hline
      &一&十&百&千&万&億&兆&京&垓&\JOcuurentChar&穣&溝\\\hline
    下数 & $10^0$  & $10^1$  & $10^2$  & $10^3$  & $10^4$  & $10^5$  & $10^6$  & $10^7$  & $10^8$  & $10^9$  & $10^{10}$ & $10^{11}$ \\\hline
    中数(万進)& $10^0$  & $10^1$  & $10^2$  & $10^3$  & $10^4$  & $10^8$  & $10^{12}$  & $10^{16}$  & $10^{20}$  & $10^{24}$  & $10^{28}$ & $10^{32}$ \\\hline
    上数& $10^0$  & $10^1$  & $10^2$  & $10^3$  & $10^4$  & $10^8$  & $10^{16}$  & $10^{32}$  & $10^{64}$  & $10^{128}$  & $10^{256}$ & $10^{512}$ \\\hline
  \end{tabular}
  \bigskip\\
  \begin{tabular}{|c|c|c|c|c|c|c|c|c|c|c|c|}\hline
     &澗&正&載&極&恒河沙&阿僧祇&那由他&不可思議&無量大数\\\hline
    下数 & $10^{12}$  & $10^{13}$  & $10^{14}$  & $10^{15}$  & $10^{16}$  & $10^{17}$  & $10^{18}$  & $10^{19}$  & $10^{20}$ \\\hline
    中数(万進)& $10^{36}$  & $10^{40}$  & $10^{44}$  & $10^{48}$  & $10^{52}$  & $10^{56}$  & $10^{60}$  & $10^{64}$  & $10^{68}$ \\\hline
    上数& $10^{1024}$  & $10^{2048}$  & $10^{4096}$  & $10^{8192}$  & $10^{16384}$  & $10^{32768}$  & $10^{65536}$  & $10^{131072}$  & $10^{262144}$ \\\hline
  \end{tabular}
\end{center}
\paragraph{上数法による表示}
「上数」による表示は下記のような入れ子構造になり、上位に下位の名称が何度も登場する。
このため表示の一部分を見た場合、そこの全体に対する位置付けが判断できず、使いにくい。
あくまで理論上考えられただけで、実務に用いられたことはないようである。
%
\begingroup%入れ子構造を見せるための \xksj@josu@c の再定義
\makeatletter
\def\xksj@josu@c#1#2#3#4{%
 \ifx #4\relax%
    \ifx #3\relax
        \xksj@fullexpandafter\@firstoffour
    \else
        \xksj@fullexpandafter\@secondoffour
    \fi
 \else
    \ifx #3\relax
        \xksj@fullexpandafter\@thirdoffour
    \else
        \xksj@fullexpandafter\@fourthoffour
    \fi
 \fi
   {#2\relax}%
   {#2{{}{#3}}\relax}%
   {\xksj@josu@c{#1}{#2}{#4}}%
   {\xksj@josu@c{#1}{#2{{}{\fbox{#4}~#1}\ {#3}}}{\relax}}%
}%
\makeatother
\fboxsep=.3ex
\begin{align*}
  &1234,1243,1324,1342,1423,1432,2134,2143\\[1ex]
  &=\text{\JoKansuji{12341243132413421423143221342143}}\\[1ex]
  &=\text{\scriptsize\JoKKansuji{12341243132413421423143221342143}}
\end{align*}
\endgroup
\paragraph{\TeX マクロ}
たとえ使いにくくとも、この表示を行なうための \TeX マクロを作る。
\begin{itembox}[l]{4桁ブロックをアラビア数字で表記する場合}
\begin{itemize}
\item \verb+\jokansuji+ \LaTeX のカウンタの表示
\item \verb+\@jokansuji+ \TeX カウンタ(\LaTeX カウンタの内部表現など)の表示
\item \verb+\JoKansuji+ アラビア数字の文字列を変換して表示
\end{itemize}  
\end{itembox}
\begin{itembox}[l]{全てを漢数字で表記する場合}
\begin{itemize}
\item \verb+\jokkansuji+ \LaTeX のカウンタの表示
\item \verb+\@jokkansuji+ \TeX カウンタ(\LaTeX カウンタの内部表現など)の表示
\item \verb+\JoKKansuji+ アラビア数字の文字列を変換して表示
\end{itemize}  
\end{itembox}
さてこのマクロを用いて先の $1234567^{89}$ を表してみよう。
\newpage
\noindent\verb+$1234567^{89}=$\JoKansuji{\xintiPow{1234567}{89}}+\par
$1234567^{89}=$
\JoKansuji{\xintiPow{1234567}{89}}\medskip\par
\noindent\verb+$1234567^{89}=$\JoKKansuji{\xintiPow{1234567}{89}}+\par
$1234567^{89}=$
\JoKKansuji{\xintiPow{1234567}{89}}\par
\begin{itembox}[l]{\bfseries クイズ}
  \centering
  上の表示の中で、最上位の漢字はどれでしょう??  
\end{itembox}
\par\medskip\noindent
「上数」を利用するなら、$(\text{無量大数})^2-1=10^{524288}-1$ までは表示できる。
もっともあまりに大きい数は、表示させようとしても \verb+TeX capacity exceeded, sorry+
となって \TeX\space が停止してしまう。
\def\thesection{第\kegonknumeral{section}話}
\section{華厳経にもとづく表示}
さて「恒河沙」以上の単位は仏典に由来する。しかしこれは単に
中国人が仏典から「借りてきた」だけであって、仏典やインド人の用法とは異なる。
そして仏典には、はるかに巨大な数が記されている。
% 「上数」と同じ考え方で実用的なものではなく、インドの実用的な体系とも異なる。
華厳経・阿僧祇品にもとづく命数法で表示してみる。
\paragraph{下位の命数法}
まず下位の数については現代インドの実用的な体系と同一である。
\begin{center}\footnotesize \hspace*{-2ex}
  \begin{tabular}{|l|c|c|c|c|c|c|c|c|c|}\hline
    & $10^0$ &    $10^1$ &    $10^2$ &    $10^3$ & $10^4$ &  $10^5$ & $10^6$  &    $10^7$ &$10^8$\\\hline
    インド英語     & one &   ten &  hundred  &  thousand & ten thousand &  lakh  & ten lakh &  crore & ten crore \\\hline
    現代ヒンディー & ek &    das &  sau  &    sahasra / haz\={a}r & das haz\={a}r &  l\={a}kh  & das l\={a}kh  & karo\d{r}  & das karo\d{r}   \\\hline
    漢訳華厳経(八十華厳) &  &  &  & &    & 洛叉 &  &  倶胝 &\\\hline
    (参考)ラーマーヤナ & eka  &  dasha & shata  & sahasra & ayuta &  lak\d{s}a  &  &  ko\d{t}i &   \\\hline
  \end{tabular}\relax  
\end{center}
インドでは「万」の数詞は用いられず 10 haz\={a}r と表現する\footnote{%
ただし上の表のラーマーヤナのように万に相当する数詞が出現する文献はある。
ここに見える$\text{ayuta}=10^4$は八十華厳の$\text{阿\YUjpChar 多}=10^{7\times2}$と同じ単語であろうか。}。
その10倍で次の数詞 100 haz\={a}r = 1 l\={a}kh が現れ、
以後同様に100倍ごとに新たな数詞が導入されるのが現代も用いられている
実務的命数法である。
たとえば 12,34,567 は 12 l\={a}kh 34 haz\={a}r 567 (12洛叉34千567)と表現することになる。

\paragraph{仏典における上位の命数法}
だが大乗仏典では宗教的な動機により「気が遠くなるほどの巨大数」を並べようとする。
この目的のために上数法で新しい数の名前を導入してゆく。たとえば漢訳仏典
\begin{quote}
  % 唐の実叉難陀(652年 - 710年)訳の80巻本(「八十華厳」、新訳、唐訳)  
  「八十華厳」(新訳、唐訳):唐の実叉難陀(652年 -- 710年)訳の80巻本
\end{quote}
の阿僧祇品では、最初に $10^2\times10^5=\text{百洛叉}=\text{倶胝}=10^7$ が述べられたあと、
その次からは上数法で数の名前が並べられる。
\[
(\text{倶胝})^2=\text{阿\YUjpChar 多}=10^{7\times2},\quad
(\text{阿\YUjpChar 多})^2=\text{那由他}=10^{7\times2^2},\quad
(\text{那由他})^2=\text{頻波羅}=10^{7\times2^3},\dots
\]
そして以下に引用するように延々と巨大数が説明される。
{\headfont\ref{appendix:八十華厳の数詞}}にその全ての指数表示を掲載する。
\begin{multicols}{3}\footnotesize \parindent=0pt
  \makeatletter
爾時心王菩薩。白佛言。世尊。諸佛如來。
演説阿僧祇無量無邊無等不可數不可稱不可思不可量不可説不可説不可説。
世尊云何。阿僧祇乃至不可説不可説耶。佛告心王菩薩言。善哉善哉。善男子。
汝今為欲令諸世間。入佛所知數量之義。而問如來應正等覺。善男子。諦聽諦聽。
善思念之。當為汝説。時心王菩薩。唯然受教。佛言。善男子。\par
一百洛叉。為一倶胝。\par
倶胝倶胝。為一阿\xksj@YU 多。\par
阿\xksj@YU 多阿\xksj@YU 多。為一那由他。\par
那由他那由他。為一頻波羅。\par
頻波羅頻波羅。為一矜羯羅。\par
矜羯羅矜羯羅。為一阿伽羅。\par
阿伽羅阿伽羅。為一最勝。\par
最勝最勝。為一摩婆羅。\par
摩婆羅摩婆羅。為一阿婆羅。\par
阿婆羅阿婆羅。為一多婆羅。\par
多婆羅多婆羅。為一界分。\par
界分界分。為一普摩。\par
普摩普摩。為一禰摩。\par
禰摩禰摩。為一阿婆\xksj@KEN 。\par
阿婆\xksj@KEN 阿婆\xksj@KEN 。為一彌伽婆。\par
彌伽婆彌伽婆。為一毘\xksj@RA 伽。\par
毘\xksj@RA 伽毘\xksj@RA 伽。為一毘伽婆。\par
毘伽婆毘伽婆。為一僧羯邏摩。\par
僧羯邏摩僧羯邏摩。為一毘薩羅。\par
毘薩羅毘薩羅。為一毘贍婆。\par
毘贍婆毘贍婆。為一毘盛伽。\par
毘盛伽毘盛伽。為一毘素陀。\par
毘素陀毘素陀。為一毘婆訶。\par
毘婆訶毘婆訶。為一毘薄底。\par
毘薄底毘薄底。為一毘\xksj@KYA 擔。\par
毘\xksj@KYA 擔毘\xksj@KYA 擔。為一稱量。\par
稱量稱量。為一一持。\par
一持一持。為一異路。\par
異路異路。為一顛倒。\par
顛倒顛倒。為一三末耶。\par
三末耶三末耶。為一毘睹羅。\par
毘睹羅毘睹羅。為一奚婆羅。\par
奚婆羅奚婆羅。為一伺察。\par
伺察伺察。為一周廣。\par
周廣周廣。為一高出。\par
高出高出。為一最妙。\par
最妙最妙。為一泥羅婆。\par
泥羅婆泥羅婆。為一訶理婆。\par
訶理婆訶理婆。為一一動。\par
一動一動。為一訶理蒲。\par
訶理蒲訶理蒲。為一訶理三。\par
訶理三訶理三。為一奚魯伽。\par
奚魯伽奚魯伽。為一達\xksj@RA 歩陀。\par
達\xksj@RA 歩陀達\xksj@RA 歩陀。為一訶魯那。\par
訶魯那訶魯那。為一摩魯陀。\par
摩魯陀摩魯陀。為一懺慕陀。\par
懺慕陀懺慕陀。為一\xksj@EI \xksj@RA 陀。\par
\xksj@EI \xksj@RA 陀\xksj@EI \xksj@RA 陀。為一摩魯摩。\par
摩魯摩摩魯摩。為一調伏。\par
調伏調伏。為一離\xksj@KYOU 慢。\par
離\xksj@KYOU 慢離\xksj@KYOU 慢。為一不動。\par
不動不動。為一極量。\par
極量極量為一阿麼怛羅。\par
阿麼怛羅阿麼怛羅。為一勃麼怛羅。\par
勃麼怛羅勃麼怛羅。為一伽麼怛羅。\par
伽麼怛羅伽麼怛羅。為一那麼怛羅。\par
那麼怛羅那麼怛羅。為一奚麼怛羅。\par
奚麼怛羅奚麼怛羅。為一\xksj@BEI 麼怛羅。\par
\xksj@BEI 麼怛羅\xksj@BEI 麼怛羅。為一鉢羅麼怛羅。\par
鉢羅麼怛羅鉢羅麼怛羅。為一尸婆麼怛羅。\par
尸婆麼怛羅尸婆麼怛羅。為一翳羅。\par
翳羅翳羅。為一薜羅。\par
薜羅薜羅。為一諦羅。\par
諦羅諦羅。為一偈羅。\par
偈羅偈羅。為一\xksj@SO 歩羅。\par
\xksj@SO 歩羅\xksj@SO 歩羅。為一泥羅。\par
泥羅泥羅。為一計羅。\par
計羅計羅。為一細羅。\par
細羅細羅。為一睥羅。\par
睥羅睥羅。為一謎羅。\par
謎羅謎羅。為一娑\xksj@RA 荼。\par
娑\xksj@RA 荼娑\xksj@RA 荼。為一謎魯陀。\par
謎魯陀謎魯陀。為一契魯陀。\par
契魯陀契魯陀。為一摩睹羅。\par
摩睹羅摩睹羅。為一娑母羅。\par
娑母羅娑母羅。為一阿野娑。\par
阿野娑阿野娑。為一迦麼羅。\par
迦麼羅迦麼羅。為一摩伽婆。\par
摩伽婆摩伽婆。為一阿怛羅。\par
阿怛羅阿怛羅。為一醯魯耶。\par
醯魯耶醯魯耶。為一薜魯婆。\par
薜魯婆薜魯婆。為一羯羅波。\par
羯羅波羯羅波。為一訶婆婆。\par
訶婆婆訶婆婆。為一毘婆羅。\par
毘婆羅毘婆羅。為一那婆羅。\par
那婆羅那婆羅。為一摩\xksj@RA 羅。\par
摩\xksj@RA 羅摩\xksj@RA 羅。為一娑婆羅。\par
娑婆羅娑婆羅。為一迷\xksj@RA 普。\par
迷\xksj@RA 普迷\xksj@RA 普。為一者麼羅。\par
者麼羅者麼羅。為一駄麼羅。\par
駄麼羅駄麼羅。為一鉢\xksj@RA 麼陀。\par
鉢\xksj@RA 麼陀鉢\xksj@RA 麼陀。為一毘迦摩。\par
毘迦摩毘迦摩。為一烏波跋多。\par
烏波跋多烏波跋多。為一演説。\par
演説演説。為一無盡。\par
無盡無盡。為一出生。\par
出生出生。為一無我。\par
無我無我。為一阿畔多。\par
阿畔多阿畔多。為一青蓮華。\par
青蓮華青蓮華。為一鉢頭摩。\par
鉢頭摩鉢頭摩。為一僧祇。\par
僧祇僧祇。為一趣。\par
趣趣。為一至。\par
至至。為一阿僧祇。\par
阿僧祇阿僧祇。為一阿僧祇轉。\par
阿僧祇轉阿僧祇轉。為一無量。\par
無量無量。為一無量轉。\par
無量轉無量轉。為一無邊。\par
無邊無邊。為一無邊轉。\par
無邊轉無邊轉。為一無等。\par
無等無等。為一無等轉。\par
無等轉無等轉。為一不可數。\par
不可數不可數。為一不可數轉。\par
不可數轉不可數轉。為一不可稱。\par
不可稱不可稱。為一不可稱轉。\par
不可稱轉不可稱轉。為一不可思。\par
不可思不可思。為一不可思轉。\par
不可思轉不可思轉。為一不可量。\par
不可量不可量。為一不可量轉。\par
不可量轉不可量轉。為一不可説。\par
不可説不可説。為一不可説轉。\par
不可説轉不可説轉。為一不可説不可説。\par
此又不可説不可説。為一不可説不可説轉。\par
爾時世尊。為心王菩薩。而説頌曰\bigskip\par
\hfil(以下略)
\makeatother
%%% Local Variables: 
%%% mode: japanese-latex
%%% TeX-master: "jousu"
%%% End: 

\end{multicols}
最後の方がイレギュラーで「不可説不可説」が2つの異なる意味で用いられている。
\begin{enumerate}\def\labelenumi{(\theenumi)}
\item $\text{不可量轉不可量轉}=(\text{不可量轉})^2=\text{不可説}$
\item $\text{不可説不可説}=(\text{不可説})(\text{不可説})=(\text{不可説})^2=\text{不可説轉}$
\item $\text{不可説轉不可説轉}=(\text{不可説轉})(\text{不可説轉})=(\text{不可説轉})^2=\text{不可説不可説}\quad\Bigl(\neq(\text{不可説})(\text{不可説})\Bigr)$
\item $\text{此又}\text{不可説不可説}=(\text{不可説不可説})(\text{不可説不可説})=(\text{不可説不可説})^2=\text{不可説不可説}\text{轉}$
\end{enumerate}
と述べられ、(2)の最左辺の「不可説不可説」は「不可説」の二乗の意味だが、
(3)の最右辺の「不可説不可説」はこれで1単位であって、「不可説」を用いるなら $(\text{不可説})^4$ である。
もしかしたら単に漢訳が不適切なだけで、サンスクリット原文では識別可能なのかも知れない。実際
\begin{quote}
  % 東晋の仏陀跋陀羅(359年 - 429年)訳の60巻本(「六十華厳」、旧訳、晋訳)  
  「六十華厳」(旧訳、晋訳):東晋の仏陀跋陀羅(359年 -- 429年)訳の60巻本
\end{quote}
においてはこの部分は
\begin{enumerate}\def\theenumi{\alph{enumi}}\def\labelenumi{(\theenumi)}
\item 不可量轉不可量轉名一不可説
\item 不可説不可説名一不可説轉
\item 不可説轉不可説轉名一不可説轉轉
\end{enumerate}
となっており識別可能である。
もっとも「六十華厳」の命数法はしょっぱなから「八十華厳」と異なっている。
\begin{itemize}
\item 百千百千名一拘梨;($=10^{5\times2}$)(つまり百千=洛叉 の次から上数法を開始)
\item 拘梨拘梨名一不變;($=10^{5\times2^2}$)
\item 不變不變名一那由他;($=10^{5\times2^3}$)
\item $\vdots$\hfil$\vdots$\hfil$\vdots$\hfil$\vdots$\hfil\hfil\hfil
\item 不可説不可説名一不可説轉;($=10^{5\times2^{120}}$)
\item 不可説轉不可説轉名一不可説轉轉。($=10^{5\times2^{121}}$)
\end{itemize}
途中に現れる数詞の漢訳名も異なっている所が多く、抜けも見られる。
「六十華厳」は「八十華厳」よりも最大数詞が小さいので、以下「八十華厳」に話を戻す。
「八十華厳」の最大数詞「不可説不可説轉」は次の値になる。
\[
\text{不可説不可説轉}=10^{7\times2^{122}}=10^{37218383881977644441306597687849648128}
\]
% この命数法によれば
% \[
% (\text{不可説不可説轉})^2-1=10^{7\times2^{123}}-1=10^{74436767763955288882613195375699296256}-1
% \]
% までの自然数を表示できる。

\paragraph{\TeX マクロ}
そこで「八十華厳」の命数法で自然数を表示する\TeX マクロを作った。
\begin{itembox}[l]{7桁ブロックをアラビア数字で表記する場合}
\begin{itemize}
\item \verb+\kegonnumeral+ \LaTeX のカウンタの表示
\item \verb+\@kegonnumeral+ \TeX カウンタ(\LaTeX カウンタの内部表現など)の表示
\item \verb+\Kegonnumeral+ アラビア数字の文字列を変換して表示
\end{itemize}  
\end{itembox}
\begin{itembox}[l]{全てを漢数字で表記する場合}
\begin{itemize}
\item \verb+\kegonknumeral+ \LaTeX のカウンタの表示
\item \verb+\@kegonknumeral+ \TeX カウンタ(\LaTeX カウンタの内部表現など)の表示
\item \verb+\Kegonknumeral+ アラビア数字の文字列を変換して表示
\end{itemize}  
\end{itembox}
なお経典の引用で旧漢字だった文字も、このマクロでは常用漢字を(対応物が存在すれば)用いている。

\paragraph{使用例}
まずはアラビア数字との混合表示から。
% 例の$1234567^{89}$は次のようになる。
% 
% \noindent\verb+$1234567^{89}=$ \Kegonnumeral{\xintiPow{1234567}{89}}+\par
% $1234567^{89}=$ \Kegonnumeral{\xintiPow{1234567}{89}}\par
% 
次の値は小さい方から16番目の\textbf{メルセンヌ素数}である。

\noindent\verb+$M_{2203}=2^{2203}-1=$\Kegonnumeral{\xintiSub{\xintiPow{2}{2203}}{1}}+
\par\noindent$M_{2203}=2^{2203}-1=$
\Kegonnumeral{\xintiSub{\xintiPow{2}{2203}}{1}}
\vfil
\noindent 次に漢字だけの表示。
次の値は小さい方から26番目の\textbf{メルセンヌ素数}である。
\newpage
\noindent\verb+$M_{23209}=2^{23209}-1=$\Kegonknumeral{\xintiSub{\xintiPow{2}{23209}}{1}}+
\par\noindent$M_{23209}=2^{23209}-1=$
\ifSkipXint
  \newread\MersenneFile \openin\MersenneFile=M23209.dat \read\MersenneFile to\TwentySixthMersennePrime \closein\MersenneFile
  \Kegonknumeral\TwentySixthMersennePrime
\else
  \Kegonknumeral{\xintiSub{\xintiPow{2}{23209}}{1}}%
\fi
\begin{center}\bfseries\large
このように実際に表示してみると、実にわかりやすい。
\end{center}
なおこの表示における最大の数詞は「$\text{多婆羅}=10^{3584}=10^{7\times2^9}$」である。
一方 $M_{23209}\approx 4.02874\times10^{6986}$ なので、「多婆羅」は表示の真中あたりに
埋もれている。

\paragraph{文字の範囲について}
\verb+\xkansuji+ では「\JOjpChar」%𥝱
の1文字を気にしていれば良かったが、今度は次のものがp\TeX の範囲外になる。
すべてBMPの文字であるが、\texttt{U+747F}はAdobe-Japan-1-6 に入っていない。
\begin{center}
  \makeatletter
  \begin{tabular}{|r|c|r|}\hline
    Unicode &  & CID(AJ)  \\\hline
    \tt U+5EBE & \xksj@YU &\tt 14508 \\\hline
    \tt U+9210 & \xksj@KEN &\tt  8647 \\\hline
    \tt U+4F49 & \xksj@KYA &\tt 14318 \\\hline
    \tt U+651E & \xksj@RA &\tt 17717 \\\hline
  % \end{tabular}
  % \hfil
  % \begin{tabular}{|r|c|r|}\hline
  %   Unicode &  & CID(AJ)  \\\hline
    \tt U+747F & \xksj@EI &\tt  なし \\\hline
    \tt U+618D & \xksj@KYOU &\tt 14552 \\\hline
    \tt U+979E & \xksj@BEI &\tt 18920 \\\hline
    \tt U+7AA3 & \xksj@SO &\tt 14930 \\\hline
  \end{tabular}
  \makeatother
\end{center}
パッケージ・オプション\verb+otf+ では、これらを\textsf{otfパッケージ}の \verb+\UTF{}+ を用いて表示する。
\bigskip

\paragraph{}
以上が\textsf{xintパッケージ}の使い方の一例である。次の\textbf{\ref{appendix:八十華厳の数詞}}も
\textsf{xintパッケージ}を利用して計算している。

\appendix
\section{八十華厳の数詞}\label{appendix:八十華厳の数詞}
\newcommand\KegonTableLines{36}\begingroup\makeatletter
\newcount\KegonToDecimalPower
\def\KegonUnitToDecimal{%
  $10^{7\times2^{\the\KegonToDecimalPower}}=10^{\xintiiMul{7}{\xintiiPow{2}{\the\KegonToDecimalPower}}}$%
  \global\advance\KegonToDecimalPower\@ne
}%
\expandafter\def\expandafter\xksj@tmp@list\expandafter{\expandafter\xksj@cdr\xksj@kgn@base@list}%
\def\xksj@tmp@unit@stop{■}%
\providecommand\KegonTableLines{36}%
\newcount\xksj@KegonTableLines \xksj@KegonTableLines=\KegonTableLines
\newcount\xksj@tempcnta
\def\xksj@make@kegon@unit@table@content{%
  \def\xksj@kegon@unit@table@content{}%
  \xksj@tempcnta=1%
  \loop%
  %%%%%%%%%%%%%%%%%%%%%%%%%%%%%%%%%%%%%%%%%%%%%%%%%%%%%%%%%%%%%%%%%%%%%%%%%%%%%%%%%%%%%%%%%%%%%%%
  \global\expandafter\expandafter\expandafter\expandafter\expandafter\expandafter\expandafter\def%
  \expandafter\expandafter\expandafter\expandafter\expandafter\expandafter\expandafter\xksj@tmp@unit%
  \expandafter\expandafter\expandafter\expandafter\expandafter\expandafter\expandafter{\xksj@car\xksj@tmp@list}%
  %%%%%%%%%%%%%%%%%%%%%%%%%%%%%%%%%%%%%%%%%%%%%%%%%%%%%%%%%%%%%%%%%%%%%%%%%%%%%%%%%%%%%%%%%%%%%%%
  \global\expandafter\def\expandafter\xksj@tmp@list\expandafter{\expandafter\xksj@cdr\expandafter{\xksj@tmp@list}}%
  %%%%%%%%%%%%%%%%%%%%%%%%%%%%%%%%%%%%%%%%%%%%%%%%%%%%%%%%%%%%%%%%%%%%%%%%%%%%%%%%%%%%%%%%%%%%%%%
  \ifx\xksj@tmp@unit\xksj@tmp@unit@stop\relax\else
  %%%%%%%%%%%%%%%%%%%%%%%%%%%%%%%%%%%%%%%%%%%%%%%%%%%%%%%%%%%%%%%%%%%%%%%%%%%%%%%%%%%%%%%%%%%%%%%
  \expandafter\expandafter\expandafter\def
  \expandafter\expandafter\expandafter\xksj@kegon@unit@table@content
  \expandafter\expandafter\expandafter{%
    \expandafter\xksj@kegon@unit@table@content\xksj@tmp@unit & \KegonUnitToDecimal\\\hline
  }%
  %%%%%%%%%%%%%%%%%%%%%%%%%%%%%%%%%%%%%%%%%%%%%%%%%%%%%%%%%%%%%%%%%%%%%%%%%%%%%%%%%%%%%%%%%%%%%%%
  \advance\xksj@tempcnta by 1%
  \ifnum\xksj@tempcnta<\xksj@KegonTableLines
  \repeat\fi
  \xksj@kegon@unit@table@content
}%
\loop
\begin{tabular}[t]{|l|l|}\hline
  \xksj@make@kegon@unit@table@content
\end{tabular}%
\quad\bigskip\hfil\vspace{\fill}\nolinebreak[0]%
  \unless\ifx\xksj@tmp@unit\xksj@tmp@unit@stop
\repeat

\makeatother\endgroup
%%% Local Variables: 
%%% mode: japanese-latex
%%% TeX-master: "jousu"
%%% End: 

\end{document}

%%% Local Variables: 
%%% mode: japanese-latex
%%% TeX-master: t
%%% End: 

